We assumed that the client and the server had a bounded memory, however the accumulation of slices and borders make the size of the history of the client and the servor grow linearly in the number of nodes added. To keep a memory bounded by a constant $M$ the following algorithm ensures that merging two processes with a small difference in their history will still be efficient but we may lose efficiency when trying to find older nodes : When adding nodes we check wether the size is greater than $M$ or not, if it is we partition the sorted list of slices in two and we remove half of the slices and according borders of the oldest half (we remove one slice out of two in the chronological order (which is the order in which the slices and borders are saved)), thus reducing the size to $\frac{3M}{4}$ and forgetting a part of the history. This algorithm will still be effective but we loose in efficiency beacause the server will assume that the client does not know some nodes it actually knew and the server will 
send them to the client. We keep the first half intact because in real-life applications most of the merging happens between the last nodes added to the history, ensuring that the algorithm will remain efficient on such merges.