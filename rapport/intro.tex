This report gives the headline and the results of my two months of internship with the NetOS (Networks and Operating Systems group) working in the Computer Lab in Cambridge. My work has been supervised by Thomas Gazagnaire. The NetOS team is included in the System Research Group of the Computer Lab. One of the project of the NetOS team is MirageOS, which is a library operating system that constructs unikernels for secure, high-performance network applications across a variety of cloud computing and mobile platforms. As a part of the MirageOS project Irmin is a Git-like distributed, branchable storage on which I have worked during my internship.
\paragraph{} The Irmin storage enables the user to synchronize distributed data structures, my contribution to the project was the elaboration of an efficient algorithm that enables an unknown number of processes doing commits to synchronize their history. The detailed algorithm produces a small (compared to the size of the history) exchange of information between a client and a server, the algorithm enables the server to be memory free regarding a particular client. 
\paragraph{} My internship began with a bibliographic work (that can be found in the bibliography 4.1) during which I had to understand the subject. An algorithm to synchronize persistant DAGs already exists and is implemented in the Git library, however my work was to find the limitations of this existing algorithm and to use the results that are known in particular cases (fixed number of processes, etc ...) in order to design an algorithm that could be used in the Irmin database. Once the algorithm designed I implemented it and tested if in OCaml.